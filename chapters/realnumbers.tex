\chapter{Real Numbers}
\section{Algebraic Objects: Fields and Order properties}
I already studied the algebraic topics before (Linear Algebra notes). So I will skip this section.
\section{Absolute value, epsilon-neighborhood}
Absolute value is a function $f: \mathbb{R} \rightarrow \mathbb{R}_0$ such that,
\begin{align*}
    &f(x) = x \qquad \text{if} \ x \ge 0 \\
    &f(x) = -x \qquad \text{if} \ x < 0
\end{align*}
Absolute value describes \textbf{Distance} between two values. It is important to think this function as distance more than some function that ``makes negative values positive''
\begin{proposition} $\forall x,y \in \mathbb{R}$,
    \begin{enumerate}
        \item $ |x| \ge 0$
        \item $|-x| = |x|$
        \item $|xy| =|x||y|$
        \item $|x|^2= x^2$
        \item $|x| \le y \iff -y \le x \le y$
        \item $-|x| \le x \le |x|$
    \end{enumerate}
    \begin{proof}
        Proofs are rather simple, so I will not bother writing here.
    \end{proof}
\end{proposition}
\begin{theorem}[\textbf{Triangle Inequality}] $\forall x,y \in \mathbb{R}$,
    \[ |x+y| \le |x| + |y| \]
    \begin{proof}
        From the proposition we have,
        \begin{align*}
            &-|x| \le x  \le |x| \\
            &-|y| \le y \le |y|
        \end{align*}
        Adding these equations we get
        \[  -|x| -|y| \le x+y \le |x| +|y| \Rightarrow |x+y| \le |x| + |y|\]
    \end{proof}
\end{theorem}
\begin{corollary} $\forall x,y \in \mathbb{R}$,
    \begin{enumerate}
        \item $ ||x| - |y|| \le |x-y|$
        \item $|x-y| \le |x| + |y|$
        \item $ \left| \sum_{i=1}^n a_i \right| \le  \sum_{i=1}^n |a_i|$
    \end{enumerate}
    \begin{proof}
        These Corollaries are direct consequence of triangle inequality, with third inequality using the proof with induction. I will not provide proofs since they are kind of boring and time comsuming.
    \end{proof}
\end{corollary}
\begin{definition}[\textbf{epsilon neighborhood}]. The $\epsilon-neighborhood$ of $a$ is defined as a set 
    \[ V_{\epsilon}(a) = \{ x \in \mathbb{R} : |x-a| < \epsilon\}\]
    Which is equivalent to open interval
    \[ (a - \epsilon, a+ \epsilon)\]
    Analysis heavily uses epsilon definitions and epsilon neighborhood for rigirous proofs. Therefore this definition is an useful tool.
\end{definition}
\section{Axiom of Completeness, Infimum and Supremum}
\begin{definition}
    A set $A \subseteq R$ is \textbf{bounded above} if $\exists b \in R$ s.t $ a \le b \ \forall a \in A$. The number $b$ is the \textbf{upper bound of  A}. We denote set of upper bounds of $A$ as $A^u$. \\
    Similarly, we define \text{lower bounds} and the set as $A^{\ell}$.
\end{definition}

\begin{definition}[\textbf{supremum}] A upper bound $a$ of a set $S$ is called \textbf{supremum} if,
    \[ a = \min A^{u}\]
Mathematically we show the notation as $a = \sup S$.\\ 
In Similar fashion, we define $b = \inf S$ for lower bounds.
\end{definition}
\vspace{6mm}
\textbf{Axiom of Completeness (AoC).} Every non-empty subsets of $\mathbb{R}$ that is bounded above have supremum. The Axiom also deduces the existence of infimum in a similar fashion.\\
\begin{lemma}[\textbf{Epsilon Definition of supremum}]
    $s \in \mathbb{R}$ is a supremum of a set $A \subseteq \mathbb{R}$ iff 
    \[ \forall \ \epsilon > 0  \exists  a \in A  |   s - \epsilon < a\] 
    \begin{proof}[Proof sketch] The both ways of the lemma can be proven by definition of the supremum.
    \end{proof}
    We use similar lemma for infimum.\\
\end{lemma}

\begin{proposition}[\textbf{Maximum and Supremum}] If maximum of $A \neq \{ \emptyset \} \subseteq \mathbb{R}$ exists, then \[ \max A = \sup A\]
\begin{proof}
    Denote $s = \sup A$ and $m = \max A$. By definition, $s = \min A^u$ and $m = A^u \cap A$. \\
    The result is an immediate consequence of the definitions of maximum and supremum.

    \vspace{4mm}
    $m$ is a proper supremum, since $\forall x\in A$ we have $x \le m$, and since also $m \in A$, $t = \sup A < m$ is impossible.
\end{proof}
Similarly, we have $\min A = \inf A$.
\end{proposition}
\begin{proposition}[\textbf{Uniqueness of Supremum}]
    Supremum and Infimum are unique.
    \begin{proof}
        For the sake of the contradiction, assume there exists two supremum $s_1, s_2$. Then by definition of supremum, we have 
        \[ s_1 \ge s_2 \ \land \ s_2 \ge s_1 \Rightarrow s_1 = s_2\]
        Infimum follows the similar proof.
    \end{proof}
\end{proposition}
\begin{proposition}[\textbf{Existence of Infimum}]
    AoC implies the existence of infimum for $A \subseteq \mathbb{R}$ such that $A^{\ell} \neq \emptyset$,
    \[\inf A = -\sup(-A)\]
    \begin{proof}
    Since $A^l \neq \emptyset$, it follows that
    \[\exists x \in A^{\ell} \ | \ x \le a\]
    Then,
    \[ -x \ge -a \Rightarrow -x \in (-A)^u \neq \emptyset \]
    By AoC, $\sup(-A)$ exists. Rest is trivial.

    \end{proof}
\end{proposition}
\begin{proposition}[\textbf{Operations on Supremum}]
    The supremum holds these properties,
    \begin{align}
        &\sup(A+B) &= \sup(A) + \sup(B) \\
        &\sup(A \cdot B) &= \sup(A) \cdot \sup(B) \\
        &\text{if} \ c \ge 0, &\sup(cA) = c\sup(A)\\
        &\text{if} \ c \le 0, &\sup(cA) = c\inf(A)
    \end{align}
    \begin{proof}
        These properties directly follow from the epsilon definition of the supremum. That is, $\forall \epsilon_a, \epsilon_b, \exists a,b \in A,B$ such that,
        \begin{align*}
            \sup(A) -a < \epsilon_a \ \land \sup(B) -b < \epsilon_b
        \end{align*}
        adding these equations to each other, we have
        \begin{align} \sup(A) + \sup(B) - (a+b) < \epsilon_a + \epsilon_b \end{align}
        Note that $(a+b) \in A+B$, and let $\epsilon_a + \epsilon_b = \epsilon_{a+b}$. Also we know that,
        \begin{align}\forall \epsilon_c \exists c \in A+B \ | \ \sup(A+B) - c < \epsilon_c \ \end{align}
        but 1.5 and 1.6 both are valid, hence the conclusion. \\
        We can similarly prove other propositions, even for inf.
    \end{proof}

\end{proposition}
\section{Applications of Completeness, Archimedean Property (A.P)}
\begin{theorem}[\textbf{Archimedean Property, A.P}] $\forall x \in \mathbb{R} \ \exists n_x \in \mathbb{N}\  | \ x \le n_x$.
    \begin{proof}
        For the sake of contradiction, assume otherwise. Then $n \le x\ \forall n \in \mathbb{N}$, by AoC $\mathbb{N}$ has supremum, $s$. Since $s-1 < s$, $s-1$ is not a upper bound, therefore $\exists m \in \mathbb{N}$ such that $s-1 < m \Rightarrow s < m+1$. but $m+1 \in \mathbb{N}$. Therefore $s$ cannot be a supremum.
    \end{proof}
\end{theorem}
\begin{theorem}[\textbf{Density of Rationals in $\mathbb{R}$}] $\forall a,b \in \mathbb{R}, \exists r \in \mathbb{Q}$ such that 
    \[ a < r < b\]
    \begin{proof} Since $r$ must be rational, we want to find $ m,n \in \mathbb{Z}$ such that $\frac{m}{n} = r$.\\
    From Archimedean property,
    \[\exists n \in \mathbb{N}: n(y-x) \ge 1\]
    Again from Archimedean property,
    \[\forall t \in \mathbb{R}, \exists m \in \mathbb{Z}: m-1 \le t \le m\]
    In other words, for any real numbers, there are two consecutive integers that lies in the each boundary of the real numbers.\\
    Let $t= nx$. Combining the inequalities, we get
    \[nx \le m \le 1+ nx \le ny \Rightarrow x \le \frac{m}{n} \le y \]
    \end{proof}
\end{theorem}
\begin{theorem}[\textbf{Density of Irrationals in $\mathbb{R}$}] $\forall x,y \in \mathbb{R}$ such that $x < y$, $\exists z \in \mathbb{I}$ such that 
    \[ x < z <y\]
    \begin{proof} It is direct consequence of density of Rationals. We apply density theorem on $\frac{x}{\sqrt{2}}$ and $\frac{y}{\sqrt{2}}$, which we will get $z = r\sqrt{2}, r \in \mathbb{Q}$, hence we are done.
    \end{proof}
\end{theorem}
\section{Intervals}
\begin{theorem}[\textbf{Closed and Open Intervals}]
    If $a,b \in \mathbb{R}$ and $a<b$, then \textbf{open interval} is defined by,
    \[ (a,b) = \{ x \in \mathbb{R}| a < x < b\}\]
    Similarly, we define \textbf{closed interval} as,
    \[ [a,b] = \{ x \in \mathbb{R}| a \le x \le b\} \]
\end{theorem}
\begin{theorem}[\textbf{Nested Intervals}]The sequence of intervals $I_n, n \in \mathbb{N}$ is nested if
    \[ I_1 \subseteq I_2 \subseteq \ldots \subseteq I_n \subseteq \ldots \]
    \end{theorem}
\begin{theorem}[\textbf{Nested Interval Property}]
    For nested intervals $\{ I_n\} = [a_n, b_n],n \in \mathbb{N}$, the below is true
        \[ \bigcap_{i=1}^{\infty}I_n \neq \emptyset \]
    \begin{proof}
        Since intervals are nested intervals, $b_1 \ge a_n \forall n \in \mathbb{N}$. Hence by AoC supremum $s$ of $\{ a_n\}$ exists.
        \\
        We know that $a_n \le s$. But since $b_n$ is also a upper bound bigger than $s$, we have $a_n \le s \le b_n$, which means $s \in  \bigcap_{i=1}^{\infty}I_n$
    \end{proof}
    \textbf{Remark:} Intervals must be closed. Consider $A_n = (0, \frac{1}{n})$. Any element of intersection must be bigger than $0$, while smaller than $\frac{1}{n}$. By Archimedian property of real numbers, this is a contradiction, hence $\bigcap_{n=1}^{\infty}A_n = \emptyset$
\end{theorem}
\section{Cardinality}
\begin{definition}[\textbf{Cardinality}]
    The sets $A,B$ have the same \textbf{cardinality} if there exists a bijective function such that $f: A \rightarrow B$. We donate cardinal equality with $A \sim B$. \\
    Cardinality mathematically describes the size of the set. 

    The $\sim$ operation is an equivalence relation.
\end{definition}
\begin{definition}[\textbf{Countable Sets}] The set $A$ is said to be \textbf{countable} if $A \sim \mathbb{N}$. Otherwise the set is called \textbf{uncountable sets}.
\end{definition}
\begin{theorem}[\textbf{Countability of $\mathbb{Q}$.}] The set $\mathbb{Q}$ is countable, that is, $\mathbb{Q} \sim \mathbb{N}$.
    \begin{proof}
        There is a proof with visual construction, which maps the rational numbers to natural numbers. 
    \end{proof}
\end{theorem}

\begin{theorem}[\textbf{Uncountability of $\mathbb{R}$}] The set $\mathbb{R}$ is uncountable.
    \begin{proof}
        Assume otherwise. Then subset $[0,1] \subseteq \mathbb{R}$ must be also countable
    \end{proof}
\end{theorem}
\begin{definition}[\textbf{Power set}] The powerset $\mathcal{P}(A)$, is the set of all subsets of $A$.
\end{definition}
\begin{theorem} Every infinite subset of a countable set is a countable set.
\end{theorem}
\begin{theorem} Let $\{ A_n\}, n = 1,2,3, \ldots$ be sequence of countable sets. Then,
    \[ S = \bigcup_{n=1}^{\infty} A_n\]
    \begin{proof} Diagonalization method (graphical)
    \end{proof}
\end{theorem}
\section{Exercises}
\begin{enumerate}
    \item* Show that for $ A = \{ 1 - \frac{1}{n} \ : \ n \in \mathbb{N}\}, \sup A = 1$.
        \begin{proof}$A$ is bounded above since clearly $\forall a \in A, a < 1$. Then by AoC,  supremum exists. Let $u = \sup A$. We will show that $ u = 1$.

            Clearly, $1$ is a upper bound, since $1 > 1 - \frac{1}{n}$ is trivial.

            if $u < 1$, we will show that there exists some $a \in A$ such that $u < a$.
            \[ \forall \epsilon > 0, \ \exists a \in A \ | \ 1 - \epsilon < a = 1 - \frac{1}{n} \Rightarrow \epsilon > \frac{1}{n}\]
            But, by Archimedean, $\exists n_0 \in \mathbb{N}$ contradicting,
            \[ u - \epsilon < 1 - \frac{1}{n} \in A\]
            Therefore $u = 1$.
        \end{proof} 
    \item If $S = \{ 1/n - 1/m : n,m \in \mathbb{N}\}$, find $\inf S$ and $\sup S$.
        \begin{proof}
            Clearly, $S$ is bounded above and below, therefore supremum and infimum exists by AoC. We will show that $\sup S = 1$, and we can find $\inf S = - \sup (-S) = -1$.  Clearly $1$ is an upper bound. By definition of supremum
            \[ \exists \epsilon > 0, \ \forall s \in S \ | \ 1 - \epsilon < s = 1/n - 1/m\Rightarrow 1 - \epsilon < 1 - \frac{1}{m}\]
            Which is equivalent to showing $\exists m \in \mathbb{N} \ | \  \epsilon > \frac{1}{m}$, which is evident from Archimedean.
        \end{proof}
    \item* Let $S$ be a set of nonnegative real numbers that is bounded above and let $T = \{ x^2 : x \in S \}$. Prove that if $u = \sup S$, then $u^2 = \sup T$.
    \begin{proof}
        Since $S$ is bounded above, $T$ is also bounded above. By AoC, supremum of $T$ exists. Let $t = \sup T$. Clearly, $u^2$ is upper bound of $T$, that is,
        \[ s \in S \ | \ s^2 \le u^2 \Rightarrow y = s^2 \in T \ | \ y \le u^2\]
        Now, we will show that $u^2$ the least upper bound, that is,
        \[ \forall \epsilon > 0 \ \exists s \in S \ | \ u^2-s^2 < \epsilon \Longrightarrow (u-s)(u+s) < \epsilon\]
        Since $u = \sup S$, we have
        \[ u -s < \epsilon_0 \ \epsilon_0 > 0\]
        Moreover, $u+s \le 2u$. Combining these inequalities, we have
        \[ (u-s)(u+s) < 2u\epsilon_0\]
        Then we just choose some $\epsilon > 2u\epsilon_0$.
    \end{proof}
    \begin{proof}[Second proof]
        \[ a = \sup A \Rightarrow a^2 = \sup A \cdot \sup A = \sup A^2 = \sup T \]
    \end{proof}
    \item Given any $x \in \mathbb{R}$, show that there exists a unique $n \in \mathbb{Z}$ such that $x \le x < n+1$.
    \begin{proof}
        By definition of floor function, we have
        \[ \lfloor x \rfloor \le x < \lfloor x \rfloor + 1\]
        Clearly, $n - \lfloor x \rfloor$ satisfies our property. Assume two $m,n \in \mathbb{Z}$ exists. WLOG $n > m$. Then,
        \[ m < n \Rightarrow m+1 \le n \Longrightarrow m+1 \le n \le x < m+1 < n+1\]
        Clearly, $m+1 < m+1$ is a contradiction.
    \end{proof}
    \item* Show that there exists $y \in \mathbb{R}$ such that $y^2 = 3$.
    \begin{proof}
        Let $S = \{ s \in \mathbb{R}: 0 \le s, s^2 < 3\}$. Clearly, $S$ is bounded, by AoC, $\sup S = u$ exists. We will show that $u^2 =3$.

        Clearly $u^2 = 3$ is an upper bound.

        If $u^2<3$, we will show that $\exists n \in \mathbb{N}: u + \frac{1}{n} \in S$
        \begin{align*}
            \left(u + \frac{1}{n} \right)^2 < 3 \Rightarrow u^2 + \frac{1}{n^2} + \frac{2u}{n} \le u^2 + \frac{1}{n}(2u + \frac{1}{n}) \Longrightarrow \frac{1}{n} < \frac{3-u^2}{2u+1}
        \end{align*}
        By Archimedean, such $n$ exists satisfying our last inequality, hence contradiction.

    \end{proof}
\item Let $I_n = [0, 1/n]$ for $n \in \mathbb{N}$. Prove that $\cap_{n=1}^{\infty}I_n = \{ 0\}$.
    \begin{proof}
        For all $n \in \mathbb{N}$, clearly $ 0 \in I_n$. For any $x > 0$, by Archimedean there exists $n_0 \in \mathbb{N}$ such that $\frac{1}{n_0} < x$, hence conclusion.
    \end{proof}
\end{enumerate}
\section{Notes and Mistakes on Exercises}
\begin{enumerate}
    \item Avoid ``intuitive'' proofs, prove every part of the proof rigorously. For example, the last exercise section, question 1, I also should prove $1 > 1 - \frac{1}{n}$ regardless of trivality.
    \item The ``steps'' in the proofs usually should be \textbf{reversed}. In a scratch paper, for example, find and construct an epsion/ natural number(?) and write it formally in the proof.
    \item Using floor function is wrong in the last exercise. A.P should be used.
\end{enumerate}
\section{References}
\begin{enumerate}
    \item \url{https://math.colorado.edu/~nita/12_Axiom_of_Completeness.pdf}
\end{enumerate}

