\chapter{Limit and Continuity}
\section{Limits of functions}
\begin{definition}[\textbf{limit}] Let $A \subset \RR$ and $c$ be a cluster point of $A$. Then, for any function $f: A \to \RR$, $L \in \RR$ is said to be a \textbf{limit of f at c.}, if $\forall \epsilon > 0$, $\exists \delta > 0$ such that if $x \in A$,
    \[ 0 < |x-c| < \delta \quad \rightarrow \quad |f(x)- L| < \epsilon \]
\end{definition}

\begin{theorem}[\textbf{Uniqueness of limit}] Limit of $f: A \to \RR$ to $c$ cluster point of $A$ is unique for $c$.
    \begin{proof}
        Assume otherwise, then two limits $L, L'$ such that $\forall \epsilon >$, $\exists \delta > 0$ that $|x-c| < \delta$ implies
        \[ |f(x) - L| < \epsilon/2\]
        \[ |f(x) - L'| < \epsilon /2\]
        Then adding them up, we have
        \[ | L - L'| \le |f(x) -L| + |f(x) - L'| < \epsilon\]
        Which gives contradiction.
    \end{proof}
\end{theorem}
\begin{theorem}[\textbf{Sequential criterion of functional limits}]
    For a function $f: A \to \RR$ and its limit point $c$ and cauchy sequence $(x_n) \subseteq A$ such that $x_n \to c$,
    \[ \limx{x}{c} f(x) = L \Leftrightarrow f(x_n) \to L\]
    \begin{proof}
        $(\Rightarrow)$ Assume that $\limx{x}{c} f(x) = L$. Then, $(\forall \epsilon > 0)(\exists \delta > 0)(\forall x \in A) $
        \[ |x-c|< \delta \Rightarrow |f(x) -L| < \epsilon\]
        $\forall (x_n)$ cauchy sequence, we know that $(\forall \delta > 0)(\exists N \in \NN)(\forall n \ge N)$
        \[ |x_n - c| < \delta\]
        But this implies that
        \[ |x_n - c| < \delta \Rightarrow |f(x_n) - L| < \epsilon\]

        $(\Leftarrow)$ For the sake of contradiction, assume $\limx{x}{c} f(x) \neq L$, then by definition, $(\exists \epsilon >0 )(\forall \delta > 0)(\forall x \in A)$,
        \[ x \in \V_{\delta}(c) \Rightarrow f(x) \not\in \Vn{L}\]
        Let this $\epsilon$ be notated as $\epsilon_0$. We construct a sequence $(x_n)$ such that for $\delta_n = \frac{1}{n}$,
        \[ x_n \in \V_{\delta_n}\]
        Then, clearly $(x_n) \rightarrow c$ since $(\delta_n) \rightarrow 0$. However,
        \[  x_n \in \V_{\delta_n} \Rightarrow f(x_n) \not\in \Vn{L}\]
        So, $f(x_n)$ does not converge to $L$, but we assumed $f(x_n) \rightarrow L$, hence contradiction.
    \end{proof}
\end{theorem}
\begin{theorem}[\textbf{Algebra operations on limit}]
    Let $A \subset \RR$, and let $f,g: A \to \RR$, and let $c \in \RR$ be a cluster point of $A$, and let $b \in \RR$.
    
    Then, similar to sequences, if $\limx{x}{c}f = L$ and $\limx{x}{c}g = M$.
    \begin{enumerate}
        \item $\limx{x}{c} (f+g) = L + M$
        \item $\limx{x}{c} (f-g) = L - M$ 
        \item $\limx{x}{c} (fg) = LM$ 
        \item $\limx{x}{c} (bf) = bL$
        \item $\limx{x}{c} (f/g) = L/M$ if $g(x) \neq 0  \forall x \in A$ and $M \neq 0$.
    \begin{proof}
        All of these statements can be proven by translating them into sequence equivalent expressions. Since we already know the algebra operations on sequence limits, we are done.
    \end{proof}
    \end{enumerate}
\end{theorem}

\begin{definition}[\textbf{Divergence for functional limits}]
    Let $f: A \to \RR$ and $c$ be a limit point of $f$. If there exists $(x_n)$ and $(y_n)$ such that
    \[ \lim x_n = \lim y_n = c \land \lim f(x_n) \neq f(y_n)\]
    We say that the functional limit at $c$ does not exist.
\end{definition}
\section{Continous Functions}
\begin{definition}[\textbf{ Continous functions}]
    A function $f: A \to \RR$ is \textbf{continous} at a limit point $c \in A$ if $\forall \epsilon >0, \exists \delta >0$ such that
    \[ |x-c|< \delta \Rightarrow |f(x) - f(c)| <\epsilon\]
    Other equivalent definitions are,
    \[\limx{x}{c}f(x) = f(c)\]
    \[ x \in V_{\delta}(c) \Rightarrow f(x) \in \Vn{f(c)}\]
    \[ (x_n) \to c \Rightarrow f(x_n) \to f(c)\]
\end{definition}

\begin{theorem}[Algebra of continous functions]
Let $f: A \to \RR$ and $g: A \to \RR$ be continous functions at $c \in A$. Then,
\begin{enumerate}
    \item $kf(x) \forall k \in \RR$ 
    \item $f(x) +g(x)$ 
    \item $f(x)g(x)$ 
    \item $f(x)/g(x), g(x) \neq 0$
\end{enumerate}
Are all continous at $c$. 
\begin{proof}
    Direct consequence of algebra of functiona limits.
\end{proof}
\end{theorem}
\begin{theorem}[\textbf{Composition of continous Functions}]
    Given functions $f,g$ where  $g(f(x))$ is well defined, if $f$ is continous and $g$ is continous at $c,f(c)$ respectively, then $g(f(x))$ is continous at $c$.
    \begin{proof}
        We use sequences to prove this theorem, let $(x_n) \to c$ be a cauchy sequence, then by continouty definition,
        \[ f(x_n) \to f(c) \Rightarrow g(f(x_n)) \to g(f(c))\]
        since $g$ is continous at $f(c)$. Hence we are done.
    \end{proof}
\end{theorem}
\begin{theorem}[\textbf{Preservation of Compact sets}] Let $f: A \to \RR$ be continous function on $A$. If $K \subset A$ is compact, then $f(K)$ is compact.
\end{theorem}
\begin{theorem} If $f:K \to \RR$ is continous on a compact set $K$, then $f$ has minimum and maximumvalues at $K$.
\end{theorem}
\begin{definition}
    A function $f:A \to \RR$ is \textbf{unfiromly continous} if $\forall \epsilon >0 \exists \delta >0$ such that 
    \[ |x-y| < \delta \Rightarrow |f(x) -f(y)| < \epsilon\]
\end{definition}
\begin{theorem} A function that is continous on a compact set $K$ is also uniformly continous at the set $K$.
\end{theorem}

\begin{theorem}[\textbf{Connectedness are preserved}] Let $f$ be continuous function at $A$. If $E \subset A$ is connected, then $f(E)$ is also connected.
\end{theorem}

\begin{definition}[\textbf{Right hand limit}] For any limit point $c$ of a set $A$ and $f$ in domain $A$, we define right hand limit as
    \[ (\forall \epsilon >0) (\exists \delta > 0)(\forall x \in A) 0 < x-c < \delta \Rightarrow |f(x) - L| < \epsilon\]
    We usually show this in this notation,
    \[ \limx{x}{c^{+}} f(x) = L\]
    Similarly we define for left hand limit
\end{definition}

\begin{theorem} For a function $f$ in domain $A$, and a limit point $c \in A$, limit of $f$ in $c$ exists iff  right hand and left hand limits are equivalent.
\end{theorem}



\begin{theorem}[\textbf{Intermediate Value Theorem}] If $f: [a,b] \to \RR$ is continuous and $L \in \RR$ lies in $(f(a), f(b))$ interval, then $\exists c\ in (a,b)$ such that $f(c) = L$.
\end{theorem}


\section{Exercises}
\begin{enumerate}
    \item Let $f \coloneqq \RR \rightarrow \RR$ and let $c \in \RR$. Show that $\underset{x \to c}{\lim} f(x) = L$ if and only if $\limx{x}{0}f(x+c) = L$.
        \begin{proof}
            By definition, $\forall \epsilon > 0$, $\exists \delta$ such that
            \[|x-c| < \delta \quad \text{means} \quad |f(x) -L| < \epsilon\]
            Choose $x \coloneqq x+c$. Then we have,
            \[ |x-0| < \delta \quad \text{means} \quad |f(x+c) - L| < \epsilon\]
        \end{proof}
    \item Let $I$ be an interval in $\RR$, let $f : I \to \RR$, and let $c \in I$. suppose $\exists K,L$ such that $|f(x) - L| \le K|x-c|$ for $x \in I$. Show that $\limx{x}{c}f(x) = L$.
        \begin{proof}
            $\forall \epsilon >0$, choose $\delta = \epsilon/K$, then
            \[ |x - c| < \epsilon/K \quad \rightarrow \quad |f(x)-L| \le K|x-c| < \epsilon\]
        \end{proof}
    \item Let $I \coloneqq (0,a)$ where $a > 0$, and let $g(x) = x^2$ for $x \in I$. For any points $x,c \in I$, show that $|g(x)-c^2| \le 2a|x-c|$. Use this inequality to prove that $\limx{x}{c}x^2 = c^2$ for any $c \in I$.
    \begin{proof}
        Since $x \in I$, $0 < x < a$. Similarly, $ 0 < c < a$. Then,
        \[ |g(x) -c^2| = |x^2-c^2| = |x-c||x+c| \le 2a|x-c|\]
        From this inequality, we choose $\delta = \epsilon/2a$. Then $\forall \epsilon > 0$,
        \[ |x-c| < \epsilon/2a \quad \rightarrow \quad |x^2-c^2|\le 2a|x-c| < \epsilon\]
    \end{proof}
\item Show that $\limx{x}{c}x^3 = c^3$ for any $c \in \RR$.
    \begin{proof}
        If $x < c$, we can choose $\delta = \epsilon/ 3c$, then $\forall \epsilon > 0$,
        \[ |x-c| < \epsilon/ 3c \quad \rightarrow \quad |x^3-c^3| = |x-c||x^2+xc+c^2| \le |x-c||3c^2| < \epsilon\]
    \end{proof}
\end{enumerate}

