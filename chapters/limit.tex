\chapter{Limit and Continuity}
\section{Limits of functions}
\begin{definition}[\textbf{limit}] Let $A \subset \RR$ and $c$ be a cluster point of $A$. Then, for any function $f: A \to \RR$, $L \in \RR$ is said to be a \textbf{limit of f at c.}, if $\forall \epsilon > 0$, $\exists \delta > 0$ such that if $x \in A$,
    \[ 0 < |x-c| < \delta \quad \rightarrow \quad |f(x)- L| < \epsilon \]
\end{definition}

\begin{theorem}[\textbf{Uniqueness of limit}] Limit of $f: A \to \RR$ to $c$ cluster point of $A$ is unique for $c$.
    \begin{proof}
        Assume otherwise, then two limits $L, L'$ such that $\forall \epsilon >$, $\exists \delta > 0$ that $|x-c| < \delta$ implies
        \[ |f(x) - L| < \epsilon/2\]
        \[ |f(x) - L'| < \epsilon /2\]
        Then adding them up, we have
        \[ | L - L'| \le |f(x) -L| + |f(x) - L'| < \epsilon\]
        Which gives contradiction.
    \end{proof}
\end{theorem}
\begin{theorem}[\textbf{Algebra operations on limit}]
    Let $A \subset \RR$, and let $f,g: A \to \RR$, and let $c \in \RR$ be a cluster point of $A$, and let $b \in \RR$.
    
    Then, similar to sequences, if $\limx{x}{c}f = L$ and $\limx{x}{c}g = M$.
    \begin{enumerate}
        \item $\limx{x}{c} (f+g) = L + M$
        \item $\limx{x}{c} (f-g) = L - M$ 
        \item $\limx{x}{c} (fg) = LM$ 
        \item $\limx{x}{c} (bf) = bL$
        \item $\limx{x}{c} (f/g) = L/M$ if $g(x) \neq 0  \forall x \in A$ and $M \neq 0$.

    \end{enumerate}
\end{theorem}



\section{Exercises}
\begin{enumerate}
    \item Let $f \coloneqq \RR \rightarrow \RR$ and let $c \in \RR$. Show that $\underset{x \to c}{\lim} f(x) = L$ if and only if $\limx{x}{0}f(x+c) = L$.
        \begin{proof}
            By definition, $\forall \epsilon > 0$, $\exists \delta$ such that
            \[|x-c| < \delta \quad \text{means} \quad |f(x) -L| < \epsilon\]
            Choose $x \coloneqq x+c$. Then we have,
            \[ |x-0| < \delta \quad \text{means} \quad |f(x+c) - L| < \epsilon\]
        \end{proof}
    \item Let $I$ be an interval in $\RR$, let $f : I \to \RR$, and let $c \in I$. suppose $\exists K,L$ such that $|f(x) - L| \le K|x-c|$ for $x \in I$. Show that $\limx{x}{c}f(x) = L$.
        \begin{proof}
            $\forall \epsilon >0$, choose $\delta = \epsilon/K$, then
            \[ |x - c| < \epsilon/K \quad \rightarrow \quad |f(x)-L| \le K|x-c| < \epsilon\]
        \end{proof}
    \item Let $I \coloneqq (0,a)$ where $a > 0$, and let $g(x) = x^2$ for $x \in I$. For any points $x,c \in I$, show that $|g(x)-c^2| \le 2a|x-c|$. Use this inequality to prove that $\limx{x}{c}x^2 = c^2$ for any $c \in I$.
    \begin{proof}
        Since $x \in I$, $0 < x < a$. Similarly, $ 0 < c < a$. Then,
        \[ |g(x) -c^2| = |x^2-c^2| = |x-c||x+c| \le 2a|x-c|\]
        From this inequality, we choose $\delta = \epsilon/2a$. Then $\forall \epsilon > 0$,
        \[ |x-c| < \epsilon/2a \quad \rightarrow \quad |x^2-c^2|\le 2a|x-c| < \epsilon\]
    \end{proof}
\item Show that $\limx{x}{c}x^3 = c^3$ for any $c \in \RR$.
    \begin{proof}
        If $x < c$, we can choose $\delta = \epsilon/ 3c$, then $\forall \epsilon > 0$,
        \[ |x-c| < \epsilon/ 3c \quad \rightarrow \quad |x^3-c^3| = |x-c||x^2+xc+c^2| \le |x-c||3c^2| < \epsilon\]
    \end{proof}
\end{enumerate}

