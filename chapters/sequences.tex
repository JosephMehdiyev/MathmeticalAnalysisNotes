\chapter{Sequences and Series}
\section{Sequences and limits}
\begin{definition}[\textbf{Sequences}] A sequence is a function with its domain as $\mathbb{N}$.
\end{definition}

\begin{definition}[\textbf{Converge}]. A sequence $(x_n)$ is said  to conerge to $x \in \mathbb{R}$, or $x$ is said to be limit of $(x_n)$ if
    \[\forall \epsilon > 0,\exists N \in \mathbb{N} \ : \ |x_n - x| < \epsilon, \ \forall n \ge N\]
    If limit exists, sequence is \textbf{convergent}, otherwise it is \textbf{divergent}.
\end{definition}

\begin{definition}[\textbf{Epsilon Neighborhood definition of convergence}]. Below definition with neighborhood is equivalent to the definition above
    \[\forall \epsilon > 0, \exists N \in \mathbb{N} \ : \ x_n \in V_{\epsilon}(x), \ \forall n > N\]
\end{definition}

\begin{theorem}[\textbf{Uniqueness of Limits}] The limit of a sequence is \textbf{unique}.
    \begin{proof}
        For the sake of the contradiction, let $x^{} = x^{'} = \lim_{n \rightarrow \infty} (x_n)$.
        with the definiton of the limit, $\forall \epsilon > 0,\exists n \in \mathbb{N}$ such that for all $n \ge N,N^{'}$,
        \[ |x - x_n| < \epsilon /2 \ \forall n \ge N\]
        \[ |x^{'} - x_n| < \epsilon / 2 \ \forall n \ge N^{'}\]
        However, by the triangle inequality, we have
        \[ |x - x^{'}| \le |x- x_n| + |x^{'} - x_n| < \epsilon/2 + \epsilon/2 = \epsilon, \ \forall n \ge K = \max (N, N^{''})\]
    Since this is $\forall \epsilon > 0$, we conclude that $x = x^{''}$.
    \end{proof}
\end{theorem}
\section{Limit Theorems}
\begin{definition} A sequence $(x_n)$ is \textbf{bounded} if there exists $U > 0$ such that
    \[ |x_n| \le U \ \forall n \in \mathbb{N}\]
    A sequence is bounded \textbf{iff} the set $\{ x_n \ : \ n \in \mathbb{N}\}$ is bounded.
\end{definition}
\begin{theorem} A convergent sequence is bounded.
    \begin{proof}
        If a sequence converges, then all but finite number of terms of the sequence belongs to $\Vn{x}$. Since $\Vnn{x}$ is bounded, the sequence itself is bounded.
    \end{proof}
\end{theorem}
\begin{theorem}[\textbf{Algebra of limits}] let $X= (x_n),Y =(y_n)$ converge to $x,y$ respectively. Then sequences $X+Y,X-Y,X \cdot Y, cX$ converge to $x+y, x-y, xy, cx$ respectively.\\
    If $y \neq 0$, $X/Y$ converges to $x/y$. \\
    \begin{proof}
        We will show that $X+Y$ property only, others are similar.
        By definition of convergence, $\forall \epsilon > 0, \exists N,N^{'} \in \mathbb{N}$ such that 
        \[|x - x_n| < \epsilon /2, \forall n \ge N\]
        \[|y - y_n| < \epsilon / 2, \forall n \ge N^{'}\]
    \end{proof}
    However, notice that $\forall n \ge \max N,N^{'}$
    \[|(x+y) - (x_n+y_n)| = |(x - x_n)+ (y-y_n)| \le |x - x_n| + |y- y_n| < \epsilon/2 + \epsilon/2 = \epsilon\]
    Which proves our theorem
\end{theorem}

\begin{theorem}
    If $(x_n)$ is convergent sequence and $x_n \ge 0$ for all $n \in \mathbb{N}$, then $x = \lim(x_n) \ge 0$.
\end{theorem}
\begin{theorem}
    if $(x_n),(y_n)$ are convergent sequences and $x_n \le y_n$ for all $n \in \mathbb{N}$, then $x \le y$.
\end{theorem}
\begin{theorem}
If $(x_n)$ is a convergent sequence and $a \le x_n \le b$ for all $n \in \mathbb{N}$, then $a \le x \le b$.
\end{theorem}
\begin{theorem}[\textbf{Squeeze theorem}]
    Let $(x_n),(y_n),(z_n)$ be sequences such that
    \[ x_n \le y_n \le z_n\]
    And $x = z$. Then $(y_n)$ converges and
    \[ x = y = z\]
\end{theorem}
All above theorems are proven similarly, the idea is the same.
\section{Monotone Sequences}
\begin{definition}
    $(x_n)$ is \textbf{monotone} if it is either increasing or decreasing.
\end{definition}
\begin{theorem}[\textbf{Monotone Convergence Theorem}] A monotone sequence is convergent iff it is bunded. Furthermore, if $x_n$ is a bounded increasing sequence, then
            \[ \lim(x_n) = \sup \{ x_n : n \in \mathbb{N}\}\]
Similarly, if $y_n$ is a bounded decreasing sequence, then
            \[ \lim(y_n) = \inf \{ y_n : n \in \mathbb{N}\}\]
\end{theorem}
\section{Subsequences}
\begin{definition}[\textbf{Subsequences}] Let $\{ n_k\}$ be strict monotone increasing sequence of real numbers, then the sequence $X^{'} = (x_{n_k})$ is called \textbf{subsequence}
\end{definition}
\begin{theorem} If a sequence $(x_n)$ converge to $x$, then the subsequence $(x_{n_k})$ also converge to $x$.
    \begin{proof}
    By definition, $\forall \epsilon >0, \  \exists N(\epsilon) \in \NN$ such that $\forall n \ge N(\epsilon)$,
    \[ |x_n -x| < \epsilon\]
    Because $n_k \ge k$ (induction), then we can find such $k \ge N(\epsilon)$, then $n_k \ge N(\epsilon)$, which means
    \[ |x_{n_K} - x| < \epsilon\]
\end{proof}
\end{theorem}

\begin{theorem}[\textbf{Monotone subsequence theorem}]
    If $(x_n)$ is a sequence, then there exists a monotone subsequence.
\end{theorem}
\begin{theorem}[\textbf{The Bolzano-Weierstrass Theorem}] A bounded sequence has a convergent subsequence.
    \begin{proof} It is direct consequence of monotone subsequence theorem. Since we can find a monotone subsequence, and is bounded, we can conclude it is convergent.
    \end{proof}
\end{theorem}
\section{The Cauchy Criterion}
\begin{definition}[\textbf{Cauchy Sequence}]
    A sequence $(x_n)$ is said to be a \textbf{Cauchy sequence} if $\forall \epsilon >0$, $\exists N(\epsilon) \in \mathbb{N}$,
    \[ |x_n -x_m| < \epsilon \quad \forall m > n > N\]
\end{definition}
\begin{theorem} A sequence is convergent if and only if it is a cauchy sequence
    \begin{proof}
    \end{proof}
\end{theorem}
\section{Exercises}
\begin{enumerate}
    \item Show that sequence of $(2^n)$ does not converge.
    \begin{proof} It suffices to prove that $(2^n)$ is unbounded. Assume otherwise that there exists $M \in \mathbb{R}$ such that $2^n \le M$ for all $n \in \mathbb{N}$. Then,
        \[ n \le \log_{2}(M) = c\]
        However by the unboundness of $\mathbb{N}$, we can find $n_0$ such that $n_0 > c$ for any $c \in \mathbb{R}$, contradicting our claim.
    \end{proof}
    \item* Show that $z_n = (a^n+b^n)^{1/n}$ where $0 < a < b$ converge to $b$.
    \begin{proof}
        Since $a >0$, we have
        \[ (a^n+b^n)^{1/n} > (b^n)^{1/n} = b\]
        Since $a < b$, we have
        \[ (a^n+b^n)^{1/n} < (2b^n)^{1/n} = 2^{1/n}b\]
        Then,
        \[b \le z_n \le 2^{1/n}b\]
        Using the squeeze theorem and the fact that $2^{1/n}$ converges to $1$, we can see that $\lim z_n = b$.
    \end{proof}
\item* Let $x_1 = 8$ and let $x_{n+1} = \frac{1}{2}x_n +2$ for $n \in \mathbb{N}$. Show that $x_n$ converges, and find the limit.
    \begin{proof} We will show that $(x_n)$ is monotone and bounded.

        1) $x_n \ge 4$ for all $n \in \mathbb{N}$.

        By induction, for $n= 1,2$ we have $8 > 4$ and $6 > 4$. Now assume it is true for $n = k$. Then,
        \[x_{k+1} = \frac{1}{2}x_k + 2 > 4\]
        2) $x_{n+1} < x_n$ for all $n \in \mathbb{N}$.
        By induction, for $n = 1,2$ we have $6 < 8$. Now assume it is true for $n = k$. Then,
        \[x_{k+1} = \frac{1}{2}x_k + 2 < \frac{1}{2} x_{k-1} + 2 = x_k\]
        Then sequence is monotone and bounded, therefore it is convergent to the $\inf\{x_n : n \in \mathbb{N}\} = 4$, which we already know how to prove.
    \end{proof}
\item Prove that $e_n = \left( 1 + \frac{1}{n} \right)^{1/n}$ is convergent.
    \begin{proof}
        Direct consequence of monotone convergence theorem.
    \end{proof}
\item* Prove that $\lim(c^{1/n}) = 1$ for $ 0 < c < 1$.
    \begin{proof}

    The sequence $(c^{1/n})$ is monotone:
    \[ c^{1/n} < c^{1/(n+1)} \Leftrightarrow \frac{1}{n} \ln c < \frac{1}{n+1} \ln c \Rightarrow \frac{1}{n} > \frac{1}{n+1} \forall n \in \mathbb{N}\]
    Which is true, since $n+1 > n \Rightarrow \frac{1}{n+1} < \frac{1}{n}$ for all natural numbers.

    The sequence is bounded:
    \[ c^{1/n} < 1 \Rightarrow c < 1\]
    Which is true since $0 < c < 1$. Then, by monotone convergence theorem, our sequence converges. Let limit be $L$. But, the subsequence $x_{2n} = c^{1/2n} = \sqrt{c^{1/n}}$ also converges to the same limit, which means
    \[ L = \sqrt{L} \Rightarrow L \in \{ 0, 1\}\]
    $L = 0$ is impossible, since $a^x = 0$ iff $a = 0$, but $0 < c$. Then, $L = 1$.
    \end{proof}
\item*  Let $(f_n)$ be the Fibonacci sequence, and let $x_n \coloneq f_{n+1}/f_n$. Given that $\lim(x_n) = L$ exists, find $L$.
    \begin{proof}
        \[ x_n = f_{n+1}/f_n = (f_n+ f_{n-1})/f_n = 1 + f_{n-1}/f_n \Rightarrow L = 1 + 1/L\]
        Solving the quadratic equation, we have $L = \frac{1}{2}(1 + \sqrt{5})$
    \end{proof}
\item * Let $(x_n)$ be a bounded sequence and for each $n \in \mathbb{N}$, let $s_n \coloneq \sup \{x_k : k \ge n \}$ and $S \coloneq \inf \{s_n \}$.
Show that there exists a subsequence of $(x_n)$ that converges to $S$.

\item * Show that the sequence $\left( \frac{n+1}{n}\right)$ is a Cauchy Sequence.
    \begin{proof}
        Choose $ M > 2/ \epsilon $, then $\forall \epsilon > 0, m > n \ge M$, $\frac{1}{m} < \frac{1}{n} \le \frac{1}{M} < \epsilon/2$, and,
        \[ \left| 1 + \frac{1}{n} - 1 - \frac{1}{m} \right| \le  \frac{1}{n} + \frac{1}{m} < \epsilon/2 + \epsilon/2 = \epsilon\]
        Which shows that our sequence is a cauchy sequence.
    \end{proof}
\item* Show that if $(x_n)$ and $(y_n)$ are cauchy sequences, then $(x_n+y_n)$ is also a cauchy sequence.
    \begin{proof}
        By definition of cauchy sequence, $\forall \epsilon > 0$,
        \[ \exists N_1 \in \mathbb{N} \ : \ |x_m - x_n| < \epsilon/2 \quad \forall m > n \ge N_1\]
        \[ \exists N_2 \in \mathbb{N} \ : \ |y_m - y_n| < \epsilon/2 \quad \forall m > n \ge N_2\]
        Choose $N = \max(N_1,N_2)$. Then for all $m > n \ge N, \epsilon > 0$,
        \[ |(x_n+y_m)-(x_m+y_m)| \le |(x_n-x_m) + (y_n+y_m)| < |x_n-x_m| + |y_n-y_m| < \epsilon /2 + \epsilon /2 = \epsilon\]
    \end{proof}
\end{enumerate}
