\chapter{Sequences and Series}
\section{Sequences and limits}
\begin{definition}[\textbf{Sequences}] A sequence is a function with its domain as $\mathbb{N}$.
\end{definition}

\begin{definition}[\textbf{Converge}]. A sequence $(x_n)$ is said  to conerge to $x \in \mathbb{R}$, or $x$ is said to be limit of $(x_n)$ if
    \[\forall \epsilon > 0,\exists N \in \mathbb{N} \ : \ |x_n - x| < \epsilon, \ \forall n \ge N\]
    If limit exists, sequence is \textbf{convergent}, otherwise it is \textbf{divergent}.
\end{definition}

\begin{definition}[\textbf{Epsilon Neighborhood definition of convergence}]. Below definition with neighborhood is equivalent to the definition above
    \[\forall \epsilon > 0, \exists N \in \mathbb{N} \ : \ x_n \in V_{\epsilon}(x), \ \forall n > N\]
\end{definition}

\begin{theorem}[\textbf{Uniqueness of Limits}] The limit of a sequence is \textbf{unique}.
    \begin{proof}
        Exercise, see exercise section
    \end{proof}
\end{theorem}
\section{Limit Theorems}
\begin{definition} A sequence $(x_n)$ is \textbf{bounded} if there exists $U > 0$ such that
    \[ |x_n| \le U \ \forall n \in \mathbb{N}\]
    A sequence is bounded \textbf{iff} the set $\{ x_n \ : \ n \in \mathbb{N}\}$ is bounded.
\end{definition}
\begin{theorem} A convergent sequence is bounded.
    \begin{proof}
        If a sequence converges, then all but finite number of terms of the sequence belongs to $\Vn{x}$. Since $\Vnn{x}$ is bounded, the sequence itself is bounded.
    \end{proof}
\end{theorem}
\begin{theorem}[\textbf{Algebra of limits}] let $X= (x_n),Y =(y_n)$ converge to $x,y$ respectively. Then sequences $X+Y,X-Y,X \cdot Y, cX$ converge to $x+y, x-y, xy, cx$ respectively.\\
    If $y \neq 0$, $X/Y$ converges to $x/y$. \\
    Proofs are Exercise.
\end{theorem}

\begin{theorem}
    If $(x_n)$ is convergent sequence and $x_n \ge 0$ for all $n \in \mathbb{N}$, then $x = \lim(x_n) \ge 0$.
\end{theorem}
\begin{theorem}
    if $(x_n),(y_n)$ are convergent sequences and $x_n \le y_n$ for all $n \in \mathbb{N}$, then $x \le y$.
\end{theorem}
\begin{theorem}
If $(x_n)$ is a convergent sequence and $a \le x_n \le b$ for all $n \in \mathbb{N}$, then $a \le x \le b$.
\end{theorem}
\begin{theorem}[\textbf{Squeeze theorem}]
    Let $(x_n),(y_n),(z_n)$ be sequences such that
    \[ x_n \le y_n \le z_n\]
    And $x = z$. Then $(y_n)$ converges and
    \[ x = y = z\]
\end{theorem}

\section{Exercises}
