\chapter{Real Numbers}
\section{Absolute value, epsilon-neighborhood}
Absolute value is a function $f: \mathbb{R} \rightarrow \mathbb{R}_0$ such that,
\begin{align*}
    & f(x) = x \qquad \text{if} \ x \ge 0 \\
    & f(x) = -x \qquad \text{if} \ x < 0
\end{align*}

Absolute value describes \textbf{Distance} between two values. It is
important to think this function as distance more than some function
that ``makes negative values positive''

\begin{prop}{Properties of absolute value}{Properties of absolute value}
    $(\forall x \in \RR),(\forall y \in \RR)$,
    \begin{enumerate}
        \item $ |x| \ge 0$
        \item $|-x| = |x|$
        \item $|xy| =|x||y|$
        \item $|x|^2= x^2$
        \item $|x| \le y \iff -y \le x \le y$
        \item $-|x| \le x \le |x|$
    \end{enumerate}
\end{prop}

\begin{theo}{Triangle Inequality}{Triangle Inequality}
    $(\forall x \in \RR)$,$(\forall y \in \RR)$,
    \[ |x+y| \le |x| + |y| \]
    \tcblower
    \begin{proof}
        From the proposition we have,
        \begin{align*}
            & -|x| \le x  \le |x| \\
            & -|y| \le y \le |y|
        \end{align*}
        Adding these equations we get
        \[  -|x| -|y| \le x+y \le |x| +|y| \Rightarrow |x+y| \le |x| + |y|\]
    \end{proof}
\end{theo}

\begin{corollary}
    $(\forall x \in \RR)$,$(\forall y \in \mathbb{R}$),
    \begin{enumerate}
        \item $ ||x| - |y|| \le |x-y|$
        \item $|x-y| \le |x| + |y|$
        \item $ \left| \sum_{i=1}^n a_i \right| \le  \sum_{i=1}^n |a_i|$
    \end{enumerate}
    \begin{proof}
        These Corollaries are direct consequence of triangle inequality,
        with third inequality using the proof with induction. I will not
        provide proofs since they are kind of boring and time comsuming.
    \end{proof}
\end{corollary}

\begin{defi}{Epsilon Neighborhood}.
    The $\epsilon-neighborhood$ of
    $a$ is defined as a set
    \[ V_{\epsilon}(a) = \{ x \in \mathbb{R} \mid |x-a| < \epsilon\}\]
    Which is equivalent to open interval
    \[ (a - \epsilon, a+ \epsilon)\]
    Analysis heavily uses epsilon definitions and epsilon neighborhood
    for rigirous proofs. Therefore this definition is an useful tool.
\end{defi}

\section{Axiom of Completeness, Infimum and Supremum}

\begin{defi}{Upper Bounds}{Upper Bounds}
    A set $A \subseteq R$ is \textbf{bounded above} if $(\exists b \in
    R)$ s.t $ a \le b \ (\forall a \in A)$. The number $b$ is the
    \textbf{upper bound of  A}. We denote set of upper bounds of $A$
    as $A^u$. \\
    Similarly, we define \text{lower bounds} and the set as $A^{\ell}$.
\end{defi}

\begin{defi}{Supremum}
    A upper bound $s$ of a set $S$
    is called supremum if,
    \[ s = \min A^{u}\]
    Mathematically we show the notation as $s = \sup S$.\\
    In Similar fashion, we define $\inf S$ for lower bounds.
\end{defi}
\vspace{6mm}

\textbf{Axiom of Completeness (AoC).} Every non-empty subsets of
$\mathbb{R}$ that is bounded above have supremum. The Axiom also
deduces the existence of infimum in a similar fashion.\\

\begin{defi}{Epsilon definition of supremum}{Epsilon definition
    of supremum}
    $s \in \mathbb{R}$ is a supremum of a set $A \subseteq \mathbb{R}$ iff
    \[ ( \forall  \epsilon > 0) (\exists  a \in A) \mid   s - \epsilon < a\]
    \begin{proof}[Proof sketch] The both ways of the lemma can be
        proven by definition of the supremum.
    \end{proof}
    We use similar lemma for infimum.\\
\end{defi}

\begin{prop}{Maximum and Supremum}{Maximum and Supremum}
    If maximum of $A  \neq \{ \emptyset \} \subseteq \mathbb{R}$
    exists, then
    \[ \max A = \sup A\]
    \begin{proof}
        Denote $s = \sup A$ and $m = \max A$. By definition, $s = \min
        A^u$ and $m = A^u \cap A$. \\
        The result is an immediate consequence of the definitions of
        maximum and supremum.

        \vspace{4mm}
        $m$ is a proper supremum, since $\forall x\in A$ we have $x \le
        m$, and since also $m \in A$, $t = \sup A < m$ is impossible.
    \end{proof}
    Similarly, we have $\min A = \inf A$.
\end{prop}

\begin{prop}{Uniqueness of Supremum}{Uniqueness of Supremum}
    Supremum and Infimum are unique.
    \begin{proof}
        For the sake of the contradiction, assume there exists two
        supremum $s_1, s_2$. Then by definition of supremum, we have
        \[ s_1 \ge s_2 \ \land \ s_2 \ge s_1 \Rightarrow s_1 = s_2\]
        Infimum follows the similar proof.
    \end{proof}
\end{prop}

\begin{prop}{Existence of Infimum}
    AoC implies the existence of infimum for $A \subseteq \mathbb{R}$
    such that $A^{\ell} \neq \emptyset$,
    \[\inf A = -\sup(-A)\]
    \begin{proof}
        Since $A^l \neq \emptyset$, it follows that
        \[(\exists x \in A^{\ell}) \ | \ x \le a\]
        Then,
        \[ -x \ge -a \Rightarrow -x \in (-A)^u \neq \emptyset \]
        By AoC, $\sup(-A)$ exists. Rest is trivial.
    \end{proof}
\end{prop}

\begin{prop}{Operations on Supremum}
    The supremum holds these properties,
    \begin{align}
        & \sup(A+B)            & = \sup(A) + \sup(B)     \\
        & \sup(A \cdot B)      & = \sup(A) \cdot \sup(B) \\
        & \text{if} \ c \ge 0, & \sup(cA) = c\sup(A)     \\
        & \text{if} \ c \le 0, & \sup(cA) = c\inf(A)
    \end{align}
    \begin{proof}
        These properties directly follow from the epsilon definition of
        the supremum. That is, $\forall \epsilon_a, \epsilon_b, \exists
        a,b \in A,B$ such that,
        \begin{align*}
            \sup(A) -a < \epsilon_a \ \land \sup(B) -b < \epsilon_b
        \end{align*}
        adding these equations to each other, we have
        \begin{align}
            \sup(A) + \sup(B) - (a+b) < \epsilon_a + \epsilon_b
        \end{align}
        Note that $(a+b) \in A+B$, and let $\epsilon_a + \epsilon_b =
        \epsilon_{a+b}$. Also we know that,
        \begin{align}
            (\forall \epsilon_c)( \exists c \in A+B) \ | \ \sup(A+B)- c < \epsilon_c \
        \end{align}
        but 1.5 and 1.6 both are valid, hence the conclusion. \\
        We can similarly prove other propositions, even for inf.
    \end{proof}
\end{prop}

\section{Applications of Completeness, Archimedean Property (A.P)}

\begin{theo}{Archimedean Property, A.P}{Archimedean Property, A.P}
    \[(\forall x \in \mathbb{R}) \ (\exists n_x \in \mathbb{N})\  | \ x \le n_x.\]
    \tcblower
    \begin{proof}
        For the sake of contradiction, assume otherwise. Then $n \le
        x\ \forall n \in \mathbb{N}$, by AoC $\mathbb{N}$ has supremum,
        $s$. Since $s-1 < s$, $s-1$ is not a upper bound, therefore
        $\exists m \in \mathbb{N}$ such that $s-1 < m \Rightarrow s <
        m+1$. but $m+1 \in \mathbb{N}$. Therefore $s$ cannot be a supremum.
    \end{proof}
\end{theo}

\begin{theo}{Density of Rationals in R}{}
    $(\forall a,b \in \mathbb{R}), (\exists r \in \mathbb{Q})$ such that
    \[ a < r < b\]
    \tcblower
    \begin{proof} Since $r$ must be rational, we want to find $ m,n \in
        \mathbb{Z}$ such that $\frac{m}{n} = r$.\\
        From Archimedean property,
        \[(\exists n \in \mathbb{N}): n(y-x) \ge 1\]
        Again from Archimedean property,
        \[(\forall t \in \mathbb{R}), (\exists m \in \mathbb{Z}): m-1 \le t \le m\]
        In other words, for any real numbers, there are two consecutive
        integers that lies in the each boundary of the real numbers.\\
        Let $t= nx$. Combining the inequalities, we get
        \[nx \le m \le 1+ nx \le ny \Rightarrow x \le \frac{m}{n} \le y \]
    \end{proof}
\end{theo}

\begin{theo}{Density of Irratioanls in R}{}
    $(\forall x,y \in \mathbb{R})$ such that $x < y$, $(\exists z \in \mathbb{I})$ such that
    \[ x < z <y\]
    \tcblower
    \begin{proof} It is direct consequence of density of Rationals. We
        apply density theorem on $\frac{x}{\sqrt{2}}$ and
        $\frac{y}{\sqrt{2}}$, which we will get $z = r\sqrt{2}, r \in
        \mathbb{Q}$, hence we are done.
    \end{proof}
\end{theo}

\section{Intervals}

\begin{theo}{Closed and Open Intervals}{}
    If $a,b \in \mathbb{R}$ and $a<b$, then \textbf{open interval} is
    defined by,
    \[ (a,b) = \{ x \in \mathbb{R}| a < x < b\}\]
    Similarly, we define \textbf{closed interval} as,
    \[ [a,b] = \{ x \in \mathbb{R}| a \le x \le b\} \]
\end{theo}

\begin{defi}{Nested Intervals}{}
    $I_n, n \in \mathbb{N}$ is nested if
    \[ I_1 \subseteq I_2 \subseteq \ldots \subseteq I_n \subseteq \ldots \]
\end{defi}

\begin{theo}{Nested Interval Property}{}
    For nested intervals $\{ I_n\} = [a_n, b_n],n \in \mathbb{N}$,
    \[ \bigcap_{i=1}^{\infty}I_n \neq \emptyset \]
    \tcblower
    \begin{proof}
        Since intervals are nested intervals, $b_1 \ge a_n \ (\forall n \in \mathbb{N})$. Hence by AoC supremum $\alpha$ of $\{ a_n\}$ exists.
        \\
        We know that $a_n \le \alpha$. But since $b_n$ is also a upper bound
        bigger than $\alpha$, we have $a_n \le \alpha \le b_n$, which means $\alpha \in
        \bigcap_{i=1}^{\infty}I_n$
    \end{proof}
    \textbf{Remark:} Intervals must be closed. Consider $A_n = (0,
    \frac{1}{n})$. Any element of intersection must be bigger than $0$,
    while smaller than $\frac{1}{n}$. By Archimedian property of real
    numbers, this is a contradiction, hence $\bigcap_{n=1}^{\infty}A_n
    = \emptyset$
\end{theo}

\section{Cardinality}

\begin{defi}{Cardinality}{}
    The sets $A,B$ have the same \textbf{cardinality} if there exists a
    bijective function such that $f: A \rightarrow B$. We donate
    cardinal equality with $A \sim B$. \\
    Cardinality mathematically describes the size of the set.

    The $\sim$ operation is an equivalence relation.
\end{defi}

\begin{defi}{Countable Sets}{}
    The set $A$ is said to be \textbf{countable} if $A \sim \mathbb{N}$. Otherwise the set is
    called \textbf{an uncountable set}.
\end{defi}

\begin{theo}{Countability of $\mathbb{Q}$}{}
    The set $\mathbb{Q}$ is countable, that is, $\mathbb{Q} \sim \mathbb{N}$.
    \begin{proof}
        There is a proof with visual construction, which maps the
        rational numbers to natural numbers.
    \end{proof}
\end{theo}

\begin{theo}{Uncountability of $\mathbb{R}$}{}
    The set $\mathbb{R}$ is uncountable.
    \begin{proof}
        Assume otherwise. Then subset $[0,1] \subseteq \mathbb{R}$ must
        be also countable
    \end{proof}
\end{theo}

\begin{defi}{Power Set}{}
    The powerset $\mathcal{P}(A)$, is the set of all subsets of $A$.
\end{defi}

\begin{theo}{}{}
    Every infinite subset of a countable set is a countable set.
\end{theo}

\begin{theo}{}{}
    Let $\{ A_n\}, n = 1,2,3, \ldots$ be sequence of
    countable sets. Then,
    \[ S = \bigcup_{n=1}^{\infty} A_n\]
    is also countable.
    \begin{proof} Diagonalization method (graphical)
    \end{proof}
\end{theo}

\section{Exercises}

\begin{problems}
\item Show that
    \[ \sup \{ x \in \RR \mid x^2 < 2 \} = \sqrt{2}\]
    \begin{proof}
        Let the set be $A$, the set is bounded, since if $(x \in A) \ x > 2 \Rightarrow x^4 > 4$, contradiction. Hence $2$
        is an upper bound.
        Let $\alpha = \sup A$. If $\alpha < 2$,
        \[(\exists n \in \NN) \mid \alpha ^2 < (\alpha +\frac{1}{n})^2 < 2 \]
        This is true because,
        \[ (\alpha + \frac{1}{n})^2 = \alpha^2 + \frac{2\alpha}{n} +\frac{1}{n^2} < \alpha^2 + \frac{2\alpha+1}{n}\]
        and by Archimedean Property,
        \[ (\exists n \in \NN) \mid \frac{\alpha^2 -2}{2\alpha + 1} > \frac{1}{n} \]
        Similarly, if $\alpha > 2$,
        \[ (\exists n \in \NN) \mid 2 < (\alpha - \frac{1}{n})^2<\alpha^2\]
        Simplifying, we get
        \[ (\alpha - \frac{1}{n})^2 = \alpha ^2 + \frac{1}{n^2} - \frac{2\alpha}{n} < \alpha ^2  - \frac{2\alpha}{n}\]
        and by Archimedean Property,
        \[(\exists n \in \NN) \mid \frac{\alpha^2 - 2}{2\alpha} > \frac{1}{n}\]
        Therefore $\alpha^2 = 2$.
    \end{proof}
\item Let $A \subset \RR$ be a nonempty set Define $-A = \{ x \mid -x \in A\}$. Show that
    \[ \sup(-A) = - \inf A.\]
    \begin{proof}
        Let $\alpha = \sup(- A)$. Suppose $-A$ is bounded above. By definition of supremum, $(\forall \epsilon > 0)(\exists x \in -A)$,
        \[ \alpha - \epsilon < x \Rightarrow - \alpha + \epsilon > -x\]
        However, $-x \in A$ and the last inequality is the definition of the infimum, hence $- \alpha = \inf A$.
        If $-A$ is not bounded
        above, then $A$ is not bounded below, hence $\sup -A = - \inf A$.
    \end{proof}
\item Let $A,B \subset \RR$ be nonepty. Define
    \[ A + B = \{ z = x+y \mid x \in A \land y \in B\}\]
    Show that
    \[ \sup (A+B) = \sup A + \sup B\]
    \begin{proof}
        If $A$ or $B$ is unbounded, then $A+B$ is unbounded. Assume Both of them are bounded.
        Let $\alpha = \sup A$, $\beta = \sup B$. Then by definition of supremum, $(\forall \epsilon > 0)(\exists a \in A)(\exists b \in B)$
        \[ \alpha - \frac{\epsilon}{2} < a\]
        \[ \beta - \frac{\epsilon}{2} < b\]
        Adding these inequalities, we get
        \[ (\alpha + \beta) - \epsilon < (a+b) \in A+B\]
        Which means $(\alpha + \beta) = \sup (A+B)$.
    \end{proof}
\item Let $A,b \subset \RR$ be nonempty. Define
    \[ A \cdot B = \{ z = x \cdot y \mid x \in A \land y \in B\}\]
    Show that
    \[ \sup (A \cdot B) = \sup A \cdot \sup B\]
    \begin{proof}
        Let $\alpha, \beta$ be supremum of $A,B$ respectively. If $A$ or $B$ is unbounded above, then $A \cdot B$ is unbounded
        above.Otherwise, by definition of supremum, $(\forall epsilon > 0)(\exists a \in A)(\exists b \in B)$
        \[ \alpha - \epsilon < a\]
        \[ \beta - \epsilon < b\]
        Multiplying these inequalities, we get,
        \[ \alpha \cdot \beta - \epsilon(\beta + \alpha - \epsilon) < a \cdot b\]
        Since $\epsilon(\beta + \alpha - \epsilon)$ is arbitrary, we see that $\alpha \cdot \beta = \sup (A \cdot B)$.
    \end{proof}
\item Let $A$ and $B$ be nonempty subsets of real numbers. Show that
    \[ \sup(A \cup B) = \max \{ \sup A, \sup B\}\]
    \begin{proof}
        Let $\alpha, \beta$ be supremum of $A,B$ respectively. If $A$ or $B$ is unbounded above, then the union is also unbounded above,
        equality is trivial. Otherwise, without loss of generality, assume $\alpha \le \beta$. Then, $(\forall x \in A \cup B), x \le
        \beta$. Since $x \in B$, we can find $b \in B$ satisfying $(\forall \epsilon > 0)$,
        \[ \beta - \epsilon > b\]
        Hence $\beta = \sup A \cup B$.
    \item
    \end{proof}
\end{problems}
\section{References}
\begin{enumerate}
    \item \url{https://math.colorado.edu/~nita/12_Axiom_of_Completeness.pdf}
\end{enumerate}
