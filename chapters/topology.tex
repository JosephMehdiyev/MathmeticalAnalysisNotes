\chapter{Basic Topology in R}
\section{Open and Closed sets}

\begin{definition}[\textbf{Open Sets}] A set is open if for $\forall a \in A \ \exists V_{\epsilon}(a) \subseteq A$.
\end{definition}

\begin{theorem} The union of open sets is open.
    \begin{proof} Let $\{ O_i: i \in I\}$ be collection of open sets and let $O = \bigcup_{i \in I} O_i$.

        $\forall a \in O_i$, we can choose such $\epsilon > 0 $ such that $V_{\epsilon}(a) \subseteq O_i$. But $O_i \subseteq O$, which implies $V_{\epsilon}(a) \subseteq O$. Since $a$ is arbitrary, we are done.
    \end{proof}
\end{theorem}
\begin{theorem} The intersection of a finite collection of open sets is open.
    \begin{proof}
        Let $\{O_i\}$ be finite collection of open sets, and let $a \in \bigcap_{i \in I} O_i$. Since these sets are open, we can find $\epsilon_1, \epsilon_2, \ldots, \epsilon_n$ such that $V_{\epsilon_i}(a) \subseteq O_i \ \forall a \in O_i$

        Then, chose $\epsilon = \min \{ e_i\}$, then
        \[ V_{\epsilon}(a) \subseteq V_{\epsilon_i}(a) \ \forall i \in I\]
        Then it follows that the intersection will contain $V_{\epsilon}(a)$, hence we are done.
    \end{proof}
\end{theorem}
\begin{definition}[\textbf{Limit points}] A point $x$ is a limit point of the set $A$ if $\forall \epsilon > 0$,
    \[ V_{\epsilon}(x) \cap A \].
    Contains some point other than $x$.

    Limit points are also called cluster points, accumulation points and so on.
\end{definition}
\begin{theorem} A point x is a limit point of the set $A$ iff $x = \lim a_n$ for some sequence $a_n \neq x$ contained in $A$.
    \begin{proof}
        $(\Rightarrow)$. $x$ is a limint point iff
        \[ (\Vn{x} \cap A) \backslash \{x\}  \neq \{ 0\}\]
        Then $\forall n \in \NN$, choose $\epsilon = 1/n$, then $\exists a_n \in A$ such that
        \[ a_n \in (V_{1/n}(x) \cap A) \backslash \{x\}\]
        Which implies $\lim a_n = x$.

        $(\Leftarrow)$. Assume a sequence $(a_n)$ exists in $A$, then $\exists N(\epsilon)$ such that $\forall n \ge N(\epsilon)$,
        \[ a_n \in \Vn{x}\]
        Then, neighborhood of $x$ contains a element distinct from $a_n$,and belongs to $A$.
    \end{proof}
\end{theorem}
\begin{definition}[\textbf{Isolated Points}] A point is called isolated point of $A$ if it is not a limit point of $A$.
\end{definition}
\begin{definition}[\textbf{Closed sets}] A set is closed if it contains its limit points.
\end{definition}
\begin{theorem} A set $A \subseteq \RR$ is closed iff every Cauchy sequence contained in $A$ has a limit that is also in $A$.
    \begin{proof}
        Since every cauchy sequence is convergent sequence, this theorem is equivalent to definition of closed sets.
    \end{proof}
\end{theorem}
\section{Closure}
\begin{definition}[\textbf{Closure}]
    Given a set $A \subseteq \RR$, let $L$ be the set of all limit points of $A$. Then closure of $A$ is defined as $\overline{A} = A \cup L$.
\end{definition}

\begin{theorem}1
    For any $A \subseteq RR$, $\overline{A}$ is the smallest closed set containing $A$.
\end{theorem}
\section{Completements}
\begin{definition}[\textbf{Complement of a set}] The complement of  a set $A \subseteq \RR$ is defined as follows,
    \[ A^c = \{x \in \RR : x \not\in A\}\]
\end{definition}

\begin{theorem}
    A set $A \subseteq \RR$ is open iff $A^c$ is closed. Consequently, $B$ is closed iff $B^c$ is open.
    \begin{proof} Exercise.
    \end{proof}
\end{theorem}
\begin{theorem} Intersection of infinitely many closed sets and union of finitely many closed sets is closed
    \begin{proof} Using the De Morgan's laws to the similar theorem of open sets, it is a direct consequence.
    \end{proof}
\end{theorem}
\section{Compact sets}
\begin{definition}[\textbf{Compact sets}]
    A set $A \subseteq \RR$ is compact if every sequence in $A$ has a subsequence that also converges in $A$.
\end{definition}
\begin{theorem}
    A set $A \subseteq \RR$ is Compact iff it is bounded and closed.
    \begin{proof}
        Assume $A$ is unbounded. Then, we can choose a sequence that is also unbounded. Then their subsequence is also unbounded. But sequences have to be bounded to converge, hence contradiction

        To prove $A$ is closed, we use the fact that subsequence converge to the same value as its sequence. But the definition requires to the value to be in $A$, hence $A$ is closed.
    \end{proof}
\end{theorem}
\section{Perfect Sets}
\begin{definition}[\textbf{Perfect sets}]
    A set is called perfect if it is closed and contains no isolated points.
\end{definition}
\begin{theorem} A nonempty perfect set is uncountable.
\end{theorem}
\section{Connected sets}
\begin{definition}
    Two non-empty sets $A,B \subseteq \RR$ are  separated if $\overline{A} \cap B$ and $A \cap \overline{B}$ are both empty. A set $C = A \cup B$ is  then called connected.
\end{definition}
\section{Exercises}
\begin{enumerate}
    \item Let $A$ be a non-empty and bounded above, so that $s = \sup A$ exists. Show that $s \in \overline{A}$.
          \begin{proof}
              By definition of supremum, $\exists a \in A$ such that
              \[ a \in \Vn{s}\]
              Which means, neighborhood of $s$ contains atleast one element, hence $s$ is a limit point. By definition of closure, $a \in \overline{A}$.
          \end{proof}
    \item  Given that $A \subseteq RR$, let $L$ be set of all limit points of $A$. Show that $L$ is closed.
          \begin{proof}
              Let $x_0$ be a limit point of $L$. Then by definition, $\forall \epsilon > 0, \exists x \in L$ such that
              \[ |x_0 - x| < \epsilon/2\]
              However, $x$ is also a limit point of $A$, then $\exists x' \in A$ such that
              \[ |x - x'| < \epsilon/2\]
              But notice that
              \[ |x_0 - x'| \le |x_0 - x| + |x - x'| < \epsilon/2 + \epsilon/2 = \epsilon\]
              Which means that $x_0$ is limit point of A, then $x_0 \in L$ for all limit points, hence we are finished.
          \end{proof}
    \item Show that if $A \subseteq$ is compact and non-empty, then $\sup A$ and $\inf A$ both exists and are elements of $A$.
          \begin{proof}
              Since $A$ is bounded, by axiom of completeness, their supremum and infimum exists. Let $s = \sup A$. Then by definition of supremum, $ \exists x \in A$ such that
              \[ x \in \Vn{s}\]
              Which means $s$ is a limit point. However, since $A$ is also closed, $s \in A$. Similarly infimum can be shown.
          \end{proof}
    \item Open cover definition is equivalent to closed and bounded definition
          \begin{proof}
              We will show that $K$ is bounded.
              Choose a set $O_x = V_{c}(x),\forall x \in K$ for some $c \in \RR$. By axiom of completeness,
              \[ \sup V_c(x), \inf V_c(x)\]
              exists. Moreover, by our assumption,
              \[ \{ O_x : x \in K\}\]
              is finite. Then,
              \[\sup \{ O_x : x \in K\} = \max \{ \sup V_c(x) : x \in K \} \]
              \[\inf \{ O_x : x \in K\} = \min \{ \inf V_c(x) : x \in K \} \]
              Therefore the set is bounded.

              Now we will show that $K$ is closed. Let $(y_n)$ be cauchy sequence such that $\forall n \in \NN, y_n \in K$. We want to show that $\lim y_n = y \in K$. For the sake of contradiction, assume $y \not\in K$. Choose sets $O_x = V_{\frac{|y-x|}{2}}(x), x \in K$. Then
              \[ \{V_{\frac{|y-x|}{2}}(x) : x \in K\} \ \text{is finite}\]
              Choose $\epsilon_0 = \min \{ \frac{|y-x|}{2}\}$
              However, by definition of convergence,
              \[ |y - y_n| < \epsilon_0 \ \forall n \ge N\]
              But it contradicts the fact that
              \[ |y- x| \le |y- y_n| + |y_n-x| < \epsilon_0 + \frac{|y-x|}{2}\le |y-x|\]
              Hence contradiction, which means $O_x$ actually does not cover our set.


          \end{proof}
\end{enumerate}
